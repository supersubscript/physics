Michaelis­Menten kinetics: explain and derive.
--------------------------------------------------
Set concentration of enzyme abiding to E_0 = E + E_B. 
Assume quasi-equilibrium for reduction of substrate (complex forming).
The system is saturated with the complex.
Set K = k1/(k_2+k_3)
Set V_{max} = E_0*k3

What’s a Hill equation?
--------------------------------------------------
In biochemistry and pharmacology, the binding of a ligand to a macromolecule is
often enhanced if there are already other ligands present on the sae
macromolecule (this is known as cooperative binding). The Hill coefficient
provides a way to quantify this effect.

It describes the fraction of the macromolecule saturated by ligand as a function
of the ligand concentration; it is used in determining the degree of
cooperativeness of the ligand binding to the enzyme or receptor.
(Ligand = funktionell grupp -- molekyl, atom, etc.) 

Has the form 1/((K_A/L)^n + 1) = L^n/(K_D + L^n).

How does it apply to transcriptional activation or repression?
--------------------------------------------------
Activation: S^n/(K_S + S^n)
Repression: 1/(K_S + S^n)

Higher n \rightarrow more abrupt trasition (in transcription factor), and
thereby determines whether the system is "on" or "off" given a certain amount of
TF.

How do you extend these equations to gene regulation by multiple transcription factors?
--------------------------------------------------
SHEA-ACKERS. Partition sums, probability of on/off. 
ASK CARL. 


Write down equations for a gene that's activated by A and repressed by B – can you come up with 
more than one solution?
--------------------------------------------------
ASK CARL.


What are advantages and disadvantages to simplifying equations? (example: the dimensionless 
equations for a bistable switch)
--------------------------------------------------
Advantages: Fewer free variables when simulating. Ease-o-tweak. 
Can reduce highly complex systems to few significant parts. 

Drawbacks: Loss of proper interpretation of mechanisms inherent to the system.
Also no rigorous interpretation of outcome, as equations are of arbitrary (no) dimension.
Loss of detail -- might overgloss important, external aspects when modelling.  

The model can (probably) be justified by comparison to actual data. 

More? 

When should you use stochastic simulations?
--------------------------------------------------
1. When you want to capture the stochasticity of a system. 
2. In investigating stability and the likes.

Describe the Gillespie algorithm.
--------------------------------------------------
1. Draw two random numbers. 
2. Increment time logarithmically. 
3. Weigh reactions according to their equational significance.
4. Normalize.
5. Split up reactions on probability line. 
6. Let random number 2 define what reaction will happen.
7. Adjust system accordingly.
8. Repeat.

Why don't we always use stochastic simulations?
--------------------------------------------------
If many reactions are present, the system can:

1. take a long, long time to simulate.
2. produce code that is not benign (too hard-coded/conditional). 
3. break system (foxes die out, and system thus requires a rescaling --
interpretation lost in translation). 
4. mustn't necessarily give other/better results than deterministic case
5. often requires a deterministic implementation for comparison. Extra work. 

Gillespie also assumes a well-stirred system. 

Can you say anything about alternatives to the basic Gillespie algorithm?
--------------------------------------------------
Modifications that don't recompute all probabilities if they haven't changed,
which is good for bigger, more complex systems.  

1. Next reaction method
Published 2000. This is an improvement over the first reaction method where 
the unused reaction times are reused. To make the sampling of reactions more 
efficient, an indexed priority queue is used to store the reaction times. On 
the other hand, to make the recomputation of propensities more efficient, a 
dependency graph is used. This dependency graph tells which reaction 
propensities to update after a particular reaction has fired.

2. Optimised and sorting direct methods
Published 2004 and 2005. These methods sort the cumulative array to reduce 
the average search depth of the algorithm. The former runs a presimulation to 
estimate the firing frequency of reactions, whereas the latter sorts the 
cumulative array on-the-fly.

3. Logarithmic direct method
Published in 2006. This is a binary search on the cumulative array, thus 
reducing the worst-case time complexity of reaction sampling to O (log M).

4. Partial-propensity methods
Published in 2009, 2010, and 2011 (Ramaswamy 2009, 2010, 2011). Use factored-out, 
partial reaction propensities to reduce the computational cost to scale with 
the number of species in the network, rather than the (larger) number of 
reactions. Four variants exist.


How would you model passive transport between cells?
--------------------------------------------------
Diffusion. 

Q: Is there a difference between transport within cells and between cells.
A: Units! 

Diffusion: length^2/time
Membrane permeability: length/time 

Also: statistical physics approach. Probabilities of moving between states. 
Chemiosmotic transport theory. 

How fast is diffusion?
--------------------------------------------------
t ~ l^2/2D (Fick's law)
Cell size ~10 $\mu m$

Travel within cell: ~3 s
Travel a distance of 1 m: ~16 years(!)

How can active transport be modelled?
--------------------------------------------------
Enzyme reaction. Many biochemical reactions are mediated by enzymes.
When between cells, remember proper indexing.  
Relation between cell A and cell B? 

What is chemical potential?
--------------------------------------------------
Describes (measures?) availability of reactant. Is defined as 
proportional to the change in entropy (or free energy) as the particle 
species number changes. 

Special cases: 
Dilute solution
Chemical equilibrium (\mu_a = \mu_b)

Can be combined with other potentials, e.g. Nernst.

What are network motifs?
--------------------------------------------------
An occurence in the interaction of elements in a network that statistically are
more significant than pure random interactions. Example: transcriptional
interaction network. Also feedforward loops, coherent ff-loops, incoherent
ff-loops.  

What dynamics are associated with some common network motifs?
--------------------------------------------------
See slides (Network motifs dynamics). 

What is a morphogen?
--------------------------------------------------
A morphogen is a substance governing the pattern of tissue development in the
process of morphogenesis, and the positions of the various specialized cell
types within a tissue. More specifically, a morphogen is a signaling molecule
that acts directly on cells to produce specific cellular responses depending on
its local concentration.

Can for example signal "Hey! Come here. Be a feather!"

What are reaction--diffusion systems? Can you give an example?
--------------------------------------------------
Systems that are keen on pattern forming. Usually includes a local, short-range part,
and a semi-global, long-range part. Example: activator--inhibitor system,
activator--substrate.

"Reaction–diffusion systems are mathematical models which explain how the
concentration of one or more substances distributed in space changes under the
influence of two processes: local chemical reactions in which the substances are
transformed into each other, and diffusion which causes the substances to spread
out over a surface in space."

What complications arise if cells aren’t all the same size?
--------------------------------------------------
What are you actually modeling? Concentrations or particle numbers? 
Do all cells produce the same amount of molecules? Are the equilibrium
concentrations the same? When adding growth -- what happens to the number of
particles, or the concentration?  

Can you say something about cell division?
--------------------------------------------------
How to model this is not currently very well known; different ideas exist.
Henrik uses methods for cell division rules that reach back to the 19th 
century:   

1. Hofmeister (1863) perpendicular to main axis of growth
2. Sachs (1878) perpendicular to old walls
3. Errara (1888) shortest path into two equal volumes (Henrik's favourite) 

And about mechanics in spatial model?
--------------------------------------------------
Stress can explain direction of growth. See Henrik's notes.  

What is the Lotka­Volterra model? What assumptions is it based on?
--------------------------------------------------
Predator--Prey.
1. Unlimited food for prey (x)
2. Predators limited only by prey
3. Predators have unlimited appetite
4. Individuals are ageless 


Can you design a more realistic model for a simple predator­prey system?
--------------------------------------------------

Consider an epidemiological model such as the SIR model. Can you interpret the parameters?
--------------------------------------------------

How would you extend the SIR model to include death? loss of resistance?
--------------------------------------------------

How could these population models be extended to include spatial aspects?
--------------------------------------------------

What problems arise when simulating spatial models stochastically?
--------------------------------------------------

What types of optimization methods can be used for fitting models to data?
--------------------------------------------------

Briefly describe the gradient based methods.
--------------------------------------------------

Briefly describe the simplex method.
--------------------------------------------------

Briefly describe simulated annealing.
--------------------------------------------------

Briefly describe evolutionary/genetic algorithms.
--------------------------------------------------

What is a cost function?
--------------------------------------------------

How could a cost function be constructed? Can you think of some alternatives?
--------------------------------------------------

What is overfitting?
--------------------------------------------------

What is sensitivity analysis, and why does it matter?
--------------------------------------------------

Why may some parameters be hard to determine?
--------------------------------------------------

How can you validate a model (and/or its parameters)?
--------------------------------------------------

What makes a model good/useful?
--------------------------------------------------
