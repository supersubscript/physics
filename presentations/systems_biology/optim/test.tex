\documentclass{beamer}
      \usepackage{lmodern}% http://ctan.org/pkg/lm
      \usepackage{float}
      \usepackage[english]{babel}
      \usepackage[utf8]{inputenc}
      \usepackage{amsmath}
      \usepackage{amssymb}
      \usepackage{color}
      \usepackage{subcaption}
      \usepackage{booktabs}
      \usepackage{tikz}
      \usepackage{multirow}
      \usetikzlibrary{decorations.pathreplacing}
      \usepackage{graphicx,epstopdf}
      \usepackage{cleveref}
      \usepackage{collcell} % loads array
      \usepackage{listings}
      \usepackage{algorithm}
      \usepackage{algpseudocode}
      \newcolumntype{m}{>{$} r <{$}}
      \newcolumntype{u}{>{$[\collectcell\si} l <{\endcollectcell]$}}
      \newcommand{\approxtext}[1]{\ensuremath{\stackrel{\text{#1}}{=}}}
      \newcommand{\matr}[1]{\mathbf{#1}}
      \newcommand{\partt}[2]{\ensuremath{\dfrac{\d {#1}}{\partial {#2}}}}
      \renewcommand{\d}[1]{\ensuremath{\operatorname{d}\!{#1}}} % non-italized differentials
      \newcommand{\h}[0]{\ensuremath{\hbar}} % hbar
      \def\changemargin#1#2{\list{}{\rightmargin#2\leftmargin#1}\item[]}
      \let\endchangemargin=\endlist 
      \usepackage{amsthm}
      \theoremstyle{plain}
      \newtheorem{thm}{theorem} % reset theorem numbering for each chapter
      \theoremstyle{definition}
      \newtheorem{defn}[thm]{definition} % definition numbers are dependent on theorem numbers
      \newtheorem{exmp}[thm]{example} % same for example numbers
      \bibliographystyle{natbib}
      \renewcommand{\theequation}{\thesection.\arabic{equation}}
      \newcommand{\ts}{\textsuperscript} 
      
      \definecolor{dkgreen}{rgb}{0,0.6,0}
      \definecolor{gray}{rgb}{0.5,0.5,0.5}
      \definecolor{mauve}{rgb}{0.58,0,0.82}

      \lstset{frame=tb,
        language=Java,
        aboveskip=3mm,
        belowskip=3mm,
        showstringspaces=false,
        columns=flexible,
        basicstyle={\small\ttfamily},
        numbers=none,
        numberstyle=\tiny\color{gray},
        keywordstyle=\color{blue},
        commentstyle=\color{dkgreen},
        stringstyle=\color{mauve},
        breaklines=true,
        breakatwhitespace=true,
        tabsize=3
      }


\begin{document}
\title{Parameter optimization using Genetic Algorithms}   
\author{Henrik Åhl} 
\date{\today} 

\frame{\titlepage} 
\section{} 

   \frame{
      \frametitle{What are Genetic Algorithms (GA)?}
   }

   \frame{
      \frametitle{What are Genetic Algorithms (GA)?}
      \begin{itemize}
            \item Heuristic search
      \end{itemize}
   }

   \frame{
      \frametitle{What are Genetic Algorithms (GA)?}
      \begin{itemize}
            \item Heuristic search
            \item Mimics natural evolution via operations such as
               \textit{mutations}, \textit{selection} and \textit{crossovers}.
      \end{itemize}
   }
   \frame{
      \frametitle{What are Genetic Algorithms (GA)?}
      \begin{itemize}
            \item Heuristic search
            \item Mimics natural evolution via operations such as
               \textit{mutations}, \textit{crossovers} and \textit{selection}. 
            \item Utilizies a set of solutions (a population) in order to
               successively produce better ones (over generations). 
      \end{itemize}
   }

\section{Method}

   \frame{
      \frametitle{Method }
      \begin{itemize}
         \item Encode all variables as \texttt{Double}s.  
      \end{itemize}
   }

   \frame{
      \frametitle{Method}
      \begin{itemize}
         \item Encode all variables as \texttt{Double}s.  
         \item Mutate a random parameter with probability $1/\#params$. \\Do this $\#params$ times. 
         \item[] \begin{center} $new value = max(a,min(N(old value,\sigma),b))$\end{center}
         \item[] where the value range is $\in [a,b]$. 
      \end{itemize}
   }
   \frame{
      \frametitle{Method}
      \begin{itemize}
         \item Encode all variables as \texttt{Double}s.  
         \item Mutate a random parameter with probability $1/\#params$. \\Do this $\#params$ times. 
         \item[] \begin{center} $new value = max(a,min(N(old value,\sigma),b))$\end{center}
         \item[] where the value range is $\in [a,b]$. 
         \item PIM-crossover. For every parameter:
         \item[] \begin{center} $new value = (1-r)\cdot parentA + r\cdot parentB$\end{center}
         \item[] where $r$ is a random number $\in [0,1]$. 
      \end{itemize}
   }

   \section{Results}
   \frame{
      \frametitle{Results -- Wildtypes}
      \centering
      \includegraphics[width=.8\textwidth]{/home/william/b16_henrikahl/optimization_images/wildtypes.png}%
   }
   \frame{
      \frametitle{Results -- Oxidation}
      \centering
      \includegraphics[width=.5\textwidth]{/home/william/b16_henrikahl/optimization_images/A-ox_mB.png}
      \includegraphics[width=.5\textwidth]{/home/william/b16_henrikahl/optimization_images/B-ox_mC.png}\\
      \includegraphics[width=.5\textwidth]{/home/william/b16_henrikahl/optimization_images/C-ox_mA.png}%
   }
   \frame{
      \frametitle{Results -- Knockout}
      \centering
      \includegraphics[width=.8\textwidth]{/home/william/b16_henrikahl/optimization_images/knockC_mA.png}%
   }

   \frame{
      \frametitle{Results -- D}
      \centering
      \includegraphics[width=.5\textwidth]{/home/william/b16_henrikahl/optimization_images/wt_mD.png}%
      \includegraphics[width=.5\textwidth]{/home/william/b16_henrikahl/optimization_images/C-ox_mD.png}%
   }
   \frame{
      \frametitle{Results -- Validation}
      \centering
      \includegraphics[width=.5\textwidth]{/home/william/b16_henrikahl/optimization_images/C-ox_mB_validation}%
      \includegraphics[width=.5\textwidth]{/home/william/b16_henrikahl/optimization_images/c-knockout_mD_validation.png}%
   }

\end{document}

