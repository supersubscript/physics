\documentclass[a4paper,12pt]{article}
   % Packages and definitions:
   % {
      \usepackage{float}
      \usepackage[english]{babel}
      \usepackage[utf8]{inputenc}
      \usepackage{amsmath}
      \usepackage{mathtools}
      \usepackage{siunitx}
% }

\begin{document}
   \section{Carl (Chapter 1-2 + handout))}
      \paragraph{Draw the DNA molecule and explain its structure and function\\}
         
         DNA is made up of the four nucleidic acids \emph{adenin, thymin,
         cytosin}, and \emph{guanine}. The structure of DNA (deoxyribonucleidic
         acid) is a so called
         \emph{polymer} -- long strings of similar units. These units, the
         nucelotides, stack neatly in an $\alpha$-helical structure. Our bodies
         contain about a meter of DNA, wound up in protein ``spools'', themselves
         forming complexes called \emph{nucleosomes}. 

         The genetic message in DNA encodes only the polypeptides primary
         structure (i.e. not how it folds), and is central in order for organisms 
         to produce the necessary proteins, components and RNA molecules. 

         The helix of DNA is made up by two strings of polymers built by the 
         different nucleotides, that by themselves might repel each other, but
         are by hydrogen bonds bound together. The four different nucleotides
         are on the other hand attached to the backbone of the polymer,
         constructed by sugar molecules and phosphate groups. Roughly every
         sugar molecule is connected to a nucleotide.   

         \vspace{5cm}

      \paragraph{Explain the flow of genetic information in the cell.\\}
         The flow of genetic information can be summarized in five steps:
         \begin{enumerate}
            \item DNA holds the blueprint for information. During cell division,
               DNA is copied by a machine called DNA polymerase. For other uses
               however, DNA contains genes that in turn contain
               \emph{regulatory} and \emph{coding} regions which specify the
               sequence of amino acids (making up proteins etc.) that are to be 
               produced. 
            \item \emph{RNA polymerase} reads DNA in a process called
               \emph{transcription}. It does so by attaching to the DNA, and
               thereafter traversing it, drawing the DNA string through a slot,
               adding successive monomers. This resulting chain is so-called
               \textbf{mRNA}. In eucaryotic cells, mRNA leaves the cell through
               pores in the membrane and enters the \emph{cytosole}. The energy
               needed to drive RNA polymerase comes from the nucleotides
               themselves, which come in high-energy NTP-form. The polymerase
               cuts of two out of three phostphate groups from each NTP as the
               nucelotide is added to the chain of transcript.
            \item In the cytosole, ribosomes bind the transcript and and walks
               along it, building up a polypeptide. This is a process called
               \emph{translation}, and builds up so-called \textbf{tRNA}
               molecules. These in turn bind to triplets of monomers and carry
               corresponding amino acid monomers to be added to the growing
               polypeptide chain. 

            \item The polypeptide either spontaneously folds into a new state,
               or does so by \textbf{chaperones} -- auxiliary proteins. 

            \item The folded protein forms a part of the cells architeture. 
         \end{enumerate}
         The overall process might be descrived by the schematic
         \begin{align*}
            Cell \rightarrow DNA \xrightarrow{transcription} mRNA \rightarrow
            ribosome \xrightarrow{translation} (tRNA \rightarrow)~protein.
         \end{align*}
               
         \vspace{3cm}
         
      \paragraph{How does the genetic code work?\\}
         DNA is made up by three different parts -- a phosphate group, a pentose
         sugar and a nitrogenous acid (e.g. adenin). The acid bases are four,
         and therefore gives rise to $4^3$ different combinations, resulting in
         the possibility of the creation of 20 different amino acids (some are
         the same even though they are built differently).

         RNA is then involved in the transcription of patterns from the DNA,
         constructing \textbf{mRNA} molecules. The pattern for protein synthesis
         is then read and translated in the ribosome, where \textbf{tRNA}thereafter adds
         amino acids to the intended sequemce. The protein is then formed, and
         can continue on to fold.

      \paragraph{How big is a human cell? A bacterium? DNA?\\}
         \begin{enumerate}
            \item [\textbf{DNA}] Width: 2 nm, Length: As long as you want it. In humans,
               about 2 m in total. Each base pair is about .5 nm, where we have
               ca 3 billion base pairs.
            
            \item [\textbf{Cell}] Volume: Sperm cell -- 30 $\si{\micro\m^3}$, red blood
               cell -- 100 $\si{\micro\m^3}$, to maximally a fat cell -- 600000 
               $\si{\micro\m^3}$, and oocytes -- 4000000 $\si{\micro\m^3}$.
            
            \item [\textbf{Bacterium}] Length: .2--.3 $\si{\micro\m^3}$, to maximally
               about $250 \si{\micro\m^3}$ long, and .75 wide. 
         \end{enumerate}

      \paragraph{How much DNA do we have in a cell?\\}
         Meterwise. 

      \paragraph{What is a gene, and how many do we have?\\}
         A gene is the region that is responsible for the actual coding of RNA
         and proteins. It is the part of the chromosome that is actually
         effectively transcribed. We have about 20000--25000. 

      \paragraph{How can different cells do different things with the same genetic
      program?\\}
         
      \paragraph{What defines the function of a protein?\\}
         Essentially the way it folds. The primary structure is pretty much the definition
         of the protein; the secondary structure describes the folding of the
         protein, resulting in the overall shape of the protein. More complex
         proteins consists of multiple polypeptide subunits that form the
         quaternary structure, e.g. hemoglobin, which has four subunits. 
         All in all, the structure of the protein define the function, since it
         determines the transport and the activation. 

      \paragraph{What functions may proteins have?\\}
         Enzymes (catalysing reactions), antibodies (binding to viri etc.), 
         messenger (e.g. hormones), structural parts (e.g. microtubuli,
         filaments), transport/storage (binds to atoms and molecules).

      \paragraph{What is the plasma membrane, and what things do we find
      there?\\}
         The plasma membrane is what protects the cells interiors from the
         exteriors. It also allows the cell to move, by being able to change
         its shape. All cells do this through the \textbf{bilayer membrane}.
         These shapes form spontaneously and are very simple in nature. It
         consists of two layers of molecules -- primarily phospholipids. It is
         only about 4 nm thick, but can cover billions of square nanometers. 

         The bilayer membrane is constructed by molecules with hydrophobic tails
         and hydrophillic heads, making it so that these layers form with heads
         pointing outward from the center of the membrane wall. 
         
      \paragraph{What structural elements help define the shape of a cell?\\}
         Macromolecular assemblies (e.g. microtubuli, actins) that form the so
         called \textbf{cytoskeleton} of the cell, in effect building up the
         different organelles for eukaryotes. Actin filaments lies under the
         surface of the cell and form a thin meshwork -- the actin cortex.
         Microvilli, filopodia and lamellipodia are full of actin-fibers which
         cross-link the one another to form stiff bundles that help to push
         these structures out of the cell. Furthermore, actin filaments form the
         ``highways'' along which single-molecule motors walk to generate muscle
         contractions and other phenomena.


      \paragraph{How can things be transported in cells?\\}
         Through binding to proteins, diffusion and drift through microtubuli.
         Also osmosis. Larger transpors are performed through \textbf{cytosis},
         where proteins form vesicles form the intended cargo to travel in.

      \paragraph{What is metabolism?\\}
         Metabolism is the process of maintaining the living state of an
         organism. It is made up of two things: 
         \begin{itemize}
            \item Anabolism -- synthesis of things needed by the cells
            \item Catabolism -- breakdown of scrap to form things that the cell
               can use
         \end{itemize}
      
      \paragraph{How is energy stored and converted in plants and animals?\\}
         Energy is stored in fats, carbohydrates and proteins in animals. In
         plants, they mainly take up sunlight and water in order to rearrange
         atoms into sugars and fats. Plants consume order -- not energy. 
      
      \paragraph{What is the function of ATP and ADP?\\}
         ATP and ADP are energy carriers. They are built up by the nitrogenous
         acid adenin and a duo- or tri-phosphate group. The mitochondria
         produces them in order to send them out as energy packages throughout
         the body. ATP contains a lot of energy due to the energy stored in the
         bond between the second and thrid phosphate group. Both ATP and ADP are
         essential for the body to be able to perform even the most basal
         functions, such as the pumping of ions through the cell membrane, or
         muscle contraction.

      \paragraph{How can complexity arise from simple building blocks?\\}
         Complexity arises from simple building blocks, or processes, due to the
         linkage of many things. 

   \section{Bosse (Chapter 3-6)}
      \subsection{Chpt. 3 - The Molecular Dance (Problems 3.1, 3.2)}
            \paragraph{
                  What does the ideal gas law say about the mean kinetic energy?
               }
               It is equal (on average) to $E_k = \frac{3}{2}kT$.
            \paragraph{
                  What does the Boltzmann distribution say in the limits of high/low T?
               }
                  High: All states equally possible. \\
                  Low: All particles in ground state.
            \paragraph{
               Explain the Arrhenius rate law.
            }
               The rate of a chemical reaction mainly depend on temperature
               through the factor $e^{-E_{barrier}}/kT$, where $E_{barrier}$ is
               some temperature independent constant inherent to the reaction.
            \paragraph{Problems}
               Problem 1: a) Mean? b) Standard deviation? \\
               Problem 2: Use that $mgh \approx \frac{mv^2}{2}$?

      \subsection{Chpt. 4 – Random Walks, Friction, and Diffusion (Problems 4.1, 4.8)}
         \paragraph{
            How does the diffusion law arise in random walk models of Brownian motion?
         }
            The mean is zero, but not the expected distance walked from the
            origin. Remember that in 2D $\langle \mathbf r^2\rangle = 4Dt$, and
            in 3D $\langle \mathbf r^2\rangle = 6Dt$. Drift? Include variance
            in step length (i.e. make it variable).
         \paragraph{
            How can one use Stokes’ law to determine the viscosity of a fluid?
         }
            Constant force: terminal velocity proportional to force.
            \begin{align*}
               D_{model} = \frac{l^2}{2\Delta t} \\
               \zeta_{model} = \frac{2m}{\Delta t}
            \end{align*}
            will imply
            \begin{align*}
               D\cdot\zeta = m(l/\Delta t)^2 = \langle mv^2 \rangle
               \stackrel{IGL}{=} kT
            \end{align*}
            Stoke's law:
            \begin{align*}
               \zeta = 6\pi\eta R
            \end{align*}
         \paragraph{
            Einstein derived a relation between the friction coefficient $\zeta$ and the diffusion constant
            D, for a body in a viscous fluid. What is so remarkable about it?
         }
            It gives an easy measure of $kT$ from macroscopic measurements.
            The relation is \emph{universal(!)}: we always get $kT$,
            independent on the type of molecule and solvent. We have no
            dependence on mass and such. Smaller molecules will experience
            less drag (i.e. less friction $\zeta$), but larger molecules will
            diffuse easier. (Not as easily halted by other particles.)  

         \paragraph{
            What is flux and what does the continuity eq. mean?
         }

            The flux is the amount of solute/substance/substrate that pushes
            through an area in a given time. The continuity equation
            essentially states that all matter going ut of a cell must go
            into a cell. There is no spontaneous loss or gain of matter.  

         \paragraph{
            Explain Fick’s law and how it leads to the Diffusion eq.
         }
            Fick's law states that the flow through a cell wall is given by the
            equation 
            \begin{align*}
               j = -D\frac{\mathop{dc}}{\mathop {dx}}
            \end{align*}
            by some constant $D$. By investigating how the concentration changes
            due to this, we get
            \begin{align*}
               \frac{\mathop {dc}}{\mathop {dt}} = -\frac{\mathop {dj}}{\mathop
               {dx}}.
            \end{align*}
            Combined we get 
            \begin{align*}
               \frac{\mathop {dc}}{\mathop {dt}} = D \frac{\mathop{d^2c}}{\mathop{dx^2}}
            \end{align*}
            which is what we call the \emph{diffusion equation}.

         \paragraph{
            Show how the Gaussian sol. for the diffusive growth of a point-like drop (of e.g. ink)
            reflects the diffusion law.
         }
            TBD


      \subsection{Chpt. 5 – Life in the Slow Lane: The Low Reynolds-Number World (Problems 5.2, 5.4)}
         \begin{enumerate}
            \item Why should one expect the atmosphere to be of the order of 104 m high?
               The concentration is proportional to the Boltzmann probability of
               a particle being there. $E \sim mg/kT$ with values gives a
               typical height $1/z^{*} = (mg/kT)^{-1} \approx 10^4$.
            \item How does a centrifuge work?
               We have a gravitational pull on the water (or liquid) equal to
               $F = V\rho g$, as well as a gravitational pull on the actual
               particle in question, so that $\Delta U = mg \Delta z - V\rho
               g\Delta z$. This all applies to a solute of some sort. 
               The sedimentation force is determined by the derivative of this
               w.r.t. position. 

               In a centrifuge the inward force is given as $f = -mr\omega^2$.
               The drift velocity must thereby be $v_{drift} = -mr\omega^2 /
               \zeta$. In equilibrium, we have $j = 0$, and so
               \begin{align*}
                  0 = D(-\frac{\partial c}{\partial r}) + r\omega^2m/kT \cdot c
               \end{align*}
               with solutions $c \sim const e^{r^2\omega^2m\beta/2}$.


            \item What is required for a situation to yield a low Reynolds number R? Give a biological
               example of the peculiarity of a low R?
         \end{enumerate}
      \subsection{Chpt. 6 – Entropy, Temperature, and Free Energy (Problems 6.2, 6.5)}
         \begin{enumerate}
            \item What does the Statistical Postulate say?
            \item What is the thermodynamical definition of temperature?
            \item State the Second Law of thermodynamics.
                A subsystem strives to minimize its (Helmholtz) Free Energy. Why is this said to
               involve a competition between entropy and energy?
            \item What is pressure, thermodynamically?
            \item A macroscopic system consists of a large number N of identical, non-interacting
               molecules, each of which can be in either of two states, labelled
               $+$ and $-$, with resp.
               energies $E_\pm = E_0 \pm \Delta E$. Show that at temperature T, the mean energy per molecule,
               E/N, will be very close to $E_0 - \Delta E \tanh (\Delta E / kT)$. (tanh is the hyperbolic tangent
               function; it is increasing and satisfies $|tanh x| < 1$.)
         \end{enumerate}



\section{Anders (Chapter 7,8,9)}

\textbf{Suggested exercises: 7.4, 7.5, 9.4, 9.5}
\subsection{Practice questions}
   \begin{enumerate}
      \item In our cylinder, forces will drive to equalize the concentration of
         our solute. Even if there are no external forces, there will be
         movement over the pistons due to the entropic forces arising when water
         can move over the membrane. We would have to apply a force equivalent
         to $\Delta p = kTc$ in order to stop the process at any given point,
         due to the van't Hoff relation, which states that a solute behaves like
         an ideal gas, and thus that the equilibrium condition for the pressure
         is given exactly by our above stated amount. 

         For every molecule, we have an individual force acting upon it with
         magnitude $f(z)$. The free energy is then given by $F(z) =
         \frac{dp}{dz} \Rightarrow F(z) = c(z)f(z)$.
   \end{enumerate}


\end{document}

